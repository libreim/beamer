\section{Otros objetos}

\begin{frame}{Listings}
  Para incluir código en las diapositivas utilizamos el paquete
  \href{https://www.ctan.org/tex-archive/macros/latex/contrib/listings}{\texttt{listings}}:
  \espacio
  \ejemplo{listing.tex}
  \espacio
  \pause
  \begin{alertblock}{Ajustando \texttt{listings}}
    Debemos incluir la opción \texttt{fragile} a las diapositivas con código.
    Además, debemos \href{http://tex.stackexchange.com/questions/24528}{extender}
    los caracteres para mostrar los que no sean ASCII.
  \end{alertblock}
\end{frame}

\begin{frame}{Enlaces internos}

\end{frame}

\begin{frame}{Gráficos}
\begin{columns}[c]
  \column{.6\textwidth}
    \begin{tikzpicture}
      \begin{axis}[
      ybar stacked,
      bar width=.7cm,
      ytick=\empty,
      symbolic x coords={\beamer,LibreOffice,PowerPoint},
      xtick=data,
      ]
        \addplot[fill=bars] coordinates
        {(\beamer,100) (LibreOffice,9) (PowerPoint,4)};
      \end{axis}
    \end{tikzpicture}

  \column{.4\textwidth}
   Podemos hacer gráficos con \href{https://ctan.org/pkg/pgfplots}{\texttt{pgfplots}} (aunque de este paquete se puede hablar tanto como de \beamer).
  \end{columns}
\end{frame}

\begin{frame}{Aún hay más...}
  \transglitter<1>
  \setbeamercovered{dynamic}
  No he podido cubrir todas las cosas que nos permite hacer \beamer \frownie{}.
  Entre otras cosas, también podemos hacer:
  \espacio
  \begin{description}[<+->]
    \item[Transiciones] Con \texttt{transglitter} obtenemos este efecto.
    \item[Multimedia] Incluir vídeos con reproductor interno (sólo en Linux) o externo.
    \item[Temporización] Ajustamos el tiempo de una diapositiva con \texttt{transsetduration}.
    \item[Animaciones] No he conseguido que funcionen \frownie{}.
    \item[Cajas]
    \begin{beamercolorbox}[shadow=true]
      Podemos poner cajas de colores para meter texto.
    \end{beamercolorbox}
    \item[Overlays de imágenes] Incluyendo, por ejemplo, dinosaurios.
  \end{description}

  \only<7>{%
  \begin{tikzpicture}[remember picture, overlay]
  \node[inner sep=0pt] at (current page.center) {%
      \includegraphics[width=\paperwidth,height=\paperheight]{./img/dinosaurio.png}%
  };%
  \end{tikzpicture}
  }
\end{frame}
