\section{Aspecto}

\subsection{Temas}
% TBD
% General, Interno, Externo, Color

\subsection{Bloques}

\begin{frame}
  \beamer permite crear bloques para estructurar la información:
  \espacio
  \begin{columns}
   \column{.5\textwidth}
      \pause
      \begin{block}{Bloques normales}
        Se crean con el entorno \\ \texttt{block}.
      \end{block}

      \pause
      \begin{alertblock}{Bloques alerta}
        Se crean con el entorno \texttt{alertblock}.
      \end{alertblock}

   \column{.5\textwidth}
    \pause
    \begin{exampleblock}{Bloques ejemplo}
      Se crean con el entorno \texttt{exampleblock}.
    \end{exampleblock}

    \pause
    \begin{theorem}
      No existen números mayores que 2.
    \end{theorem}

  \end{columns}

  \pause
  \espacio
  Los bloques matemáticos (teoremas, demostraciones...) tienen el mismo aspecto
  que los bloques normales.
\end{frame}

\subsection{Overlays}

\frame{
  \frametitle{\textit{Overlays}}
  \framesubtitle{Entornos de enumeración}
  Los \textit{overlays} permiten mostrar elementos selectivamente.
  Pueden utilizarse en casi cualquier elemento de \beamer.

  \espacio

  \begin{exampleblock}{\textit{Overlays} en \texttt{itemize}}

  \begin{columns}
    \column{.3\textwidth}
      \muestra{overlay.tex}
    \column{.5\textwidth}
      \ejemplo{overlay.tex}
  \end{columns}

  \end{exampleblock}
}

\frame{
  \frametitle{\textit{Overlays}}
  \framesubtitle{Formato y matemáticas}
  % TBD
}

\frame{
\frametitle{\textit{Overlays} en \texttt{tikz}}
\framesubtitle{Copiado vilmente de
\href{http://tex.stackexchange.com/questions/55806}{tex.stackexchange.com}}
\begin{center}
  \begin{tikzpicture}[mindmap, concept color=gray!50, font=\sf, text=white]
    \tikzstyle{level 1 concept}+=[font=\sf, sibling angle=90,level distance = 30mm]
    \node[concept,scale=0.7] {También podemos}
    [clockwise from=135]
    child[concept color=orange, visible on=<2->]{ node[concept,scale=0.7]{usar los} }
    child[concept color=orange, visible on=<3->]{ node[concept,scale=0.7]{\textit{overlays}} }
    child[concept color=orange, visible on=<4->]{ node[concept,scale=0.7]{en} }
    child[concept color=orange, visible on=<5->]{ node[concept,scale=0.7]{\texttt{tikz}} };
  \end{tikzpicture}
\end{center}
}

\subsection{Columnas} %TBD

\subsection{Formato}

\frame{
\frametitle{Tamaño y color}
\begin{columns}
  \column{.5\textwidth}
    Podemos cambiar el tamaño de letra utilizando los comandos habituales en $\LaTeX$.
    También podemos cambiar el color, utilizando el paquete \texttt{xcolor}.
    \espacio
    Los colores básicos son: {\color{white} blanco}, {\color{black} negro},
    {\color{red} rojo}, {\color{green} verde}, {\color{blue} azul},
    {\color{cyan} cian}, {\color{magenta} magenta} y {\color{yellow} amarillo},
    aunque se pueden \href{http://en.wikibooks.org/wiki/LaTeX/Colors}{ampliar y combinar}.

  \column{.5\textwidth}
    \begin{itemize}
      \item \texttt{\tiny \textbackslash tiny}
      \item \texttt{\scriptsize \textbackslash scriptsize}
      \item \texttt{\footnotesize \textbackslash footnotesize}
      \item \texttt{\small \textbackslash small}
      \item \texttt{\normalsize \textbackslash normalsize}
      \item \texttt{\large \textbackslash large}
      \item \texttt{\Large \textbackslash Large}
      \item \texttt{\LARGE \textbackslash LARGE}
      \item \texttt{\huge \textbackslash huge}
    \end{itemize}
\end{columns}
}

\frame{
\frametitle{Matemáticas}

Como en cualquier documento de $\LaTeX$, podemos mostrar expresiones matemáticas
con la sintaxis habitual:
  \[\mathcal{L} = \frac{-1}{4} F^2  + i\bar{\psi}D\!\!\!\!/\ \psi
  + \bar{\psi}\phi\psi  + h.c.  + |D\phi|^2 - V (\phi)\]

\pause
\begin{alertblock}{Cambiando el tipo de letra}
  \beamer utiliza una letra sin serifa para las fórmulas matemáticas por defecto.
  Podemos utilizar la fuente con serifa
  \href{http://tex.stackexchange.com/questions/34265}{incluyendo}:

  \texttt{{\color{black}\textbackslash}{\color{keywords}usefonttheme}{\color{black} [onlymath]\{serif\}}}
\end{alertblock}
}
