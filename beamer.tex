%% Paquetes y configuración %

% Beamer
\documentclass{beamer}

% Idioma
\usepackage[spanish]{babel} % Traducciones
\usepackage[utf8]{inputenc} % Uso de caracteres UTF-8

% Matemáticas
\usepackage{amsfonts}
\usepackage{amsmath}
\usepackage{amssymb}

% Colores
\usepackage{color}
\definecolor{backg}{HTML}{F2F2F2}    % Strings
\definecolor{comments}{HTML}{BDBDBD} % Comentarios
\definecolor{keywords}{HTML}{8A0808} % Palabras clave
\definecolor{strings}{HTML}{FA5858}  % Strings
\definecolor{links}{HTML}{DF0101}    % Enlaces
\definecolor{bars}{HTML}{B40404}     % Barras (gráfico)

% Código
\usepackage{listings}
\lstset{
%frame=shadowbox,
language=[LaTeX]TeX,
basicstyle=\footnotesize,
morekeywords={frametitle,framesubtitle, href},
breaklines=true,
backgroundcolor=\color{backg},
keywordstyle=\color{keywords},
commentstyle=\color{comments},
stringstyle=\color{strings},
tabsize=2,
% Incluir símbolos no ASCII (tex.stackexchange.com/questions/24528)
extendedchars=true,
literate={á}{{\'a}}1 {é}{{\'e}}1 {í}{{\'i}}1 {ó}{{\'o}}1  {ú}{{\'u}}1 {¡}{{!`}}1 {¿}{{?`}}1}
}

% Gráficos
\usepackage{pgfplots}
\pgfplotsset{width=7cm,compat=1.8} % Opciones para gráficos


%% Comandos %%
\newcommand{\iffd}{\overset{\Delta}{\iff}} % Si y sólo si por definición
\newcommand{\N}{\mathbb{N}}                % Números naturales
\newcommand{\R}{\mathbb{R}}              % Números reales
\newcommand{\C}{\mathbb{C}}              % Números complejos
\newcommand{\Power}{\mathcal{P}}           % Conjunto potencia
\newcommand{\st}{\mathrel{} : \mathrel{}}  % Tal que (dos puntos)


%% Temas %%
% Tema y tema de color
\usetheme{Dresden}
\usecolortheme{beaver}


% Enlaces (tex.stackexchange.com/questions/13423)
\hypersetup{colorlinks,linkcolor=,urlcolor=links}


%% Título y otros %%
\title{Cómo usar \texttt{beamer}}  % Título
\subtitle{Una guía escrita en \texttt{beamer}}   % Subtítulo
\author[Pablo Baeyens]{Pablo Baeyens Fernández \\ \href{mailto:pbaeyens31+github@gmail.com}{pbaeyens31+github@gmail.com}} %Autor y e-mail
\date{DGIIM} % Fecha


%% Presentación %%
\begin{document}

% Página de título
\frame{\titlepage}

\frame{
  \frametitle{Índice}
  \tableofcontents[currentsection]
}

\section{Introducción}

\frame{
\frametitle{¡Contribuye!}
  El código fuente de éstas diapositivas está disponible en:
\begin{center}
  \huge \href{http://github.com/pbaeyens/beamer}{github.com/pbaeyens/beamer}
\end{center}
  Erratas, correcciones y aportaciones son bienvenidas.
}


\frame{
  \frametitle{Primeros pasos}
  Para empezar a usar \texttt{beamer}, indicamos la clase del documento:

  \lstinputlisting[firstline=4, lastline=4]{beamer.tex}

  Podemos utilizar cualquier\footnote{\texttt{pandoc} tiene una sintaxis \href{http://johnmacfarlane.net/pandoc/demo/example9/producing-slide-shows-with-pandoc}{diferente}.} compilador de $\LaTeX$ (yo he utilizado \href{http://www.ctan.org/pkg/texliveonfly}{\texttt{texliveonfly}}).
}

\frame{
  En el entorno \texttt{document} creamos las diapositivas utilizando el
  entorno \texttt{frame}:

  \lstinputlisting{ejemplo1.tex}

  También podemos utilizar \texttt{frame} como una función.
}

\begin{frame}
  \frametitle{Título}
  \framesubtitle{Subtítulo}
  Aquí va el texto.
\end{frame}


\frame{}
% Secciones
\section{Gráficos y otras maravillas de \texttt{tikz}}

\frame{
\begin{columns}[c]
  \column{.6\textwidth}
    \begin{tikzpicture}
      \begin{axis}[
      ybar stacked,
      bar width=.7cm,
      ytick=\empty,
      symbolic x coords={\texttt{beamer},LibreOffice,PowerPoint},
      xtick=data,
      ]
        \addplot[fill=bars] coordinates
        {(\texttt{beamer},100) (LibreOffice,9) (PowerPoint,4)};
      \end{axis}
    \end{tikzpicture}

  \column{.4\textwidth}
   Podemos hacer gráficos con \href{https://ctan.org/pkg/pgfplots}{\texttt{pgfplots}} (aunque de este paquete se puede hablar tanto como de \texttt{beamer}).
  \end{columns}
}
\end{document}
