% Autor: Pablo Baeyens (@pbaeyens)
% Email: pbaeyens31+github@gmail.com
% Licencia: CC BY-SA 3.0

%% Paquetes y configuración %

% Beamer
\documentclass{beamer}

% Idioma
\usepackage[spanish]{babel} % Traducciones
\usepackage[utf8]{inputenc} % Uso de caracteres UTF-8
\uselanguage{Spanish}       % Traducciones beamer
\languagepath{Spanish}      % tex.stackexchange.com/questions/168208

% Matemáticas
\usepackage{amsfonts}
\usepackage{amsmath}
\usepackage{amssymb}

% Colores
\definecolor{backg}{HTML}{F2F2F2}    % Fondo
\definecolor{title}{HTML}{bdc3d1}    % Títulos
\definecolor{comments}{HTML}{BDBDBD} % Comentarios
\definecolor{keywords}{HTML}{08388c} % Palabras clave
\definecolor{strings}{HTML}{FA5858}  % Strings
\definecolor{links}{HTML}{3333B2}    % Enlaces
\definecolor{bars}{HTML}{045FB4}     % Barras (gráfico)

% Código
\usepackage{listings}
\lstset{
language=[LaTeX]TeX,
basicstyle=\footnotesize,
morekeywords={frametitle,framesubtitle, href},
breaklines=true,
backgroundcolor=\color{backg},
keywordstyle=\color{keywords},
commentstyle=\color{comments},
stringstyle=\color{strings},
tabsize=2,
% Incluir símbolos no ASCII (tex.stackexchange.com/questions/24528)
extendedchars=true,
literate={á}{{\'a}}1 {é}{{\'e}}1 {í}{{\'i}}1 {ó}{{\'o}}1
         {ú}{{\'u}}1 {ñ}{{\~n}}1 {¡}{{\textexclamdown}}1
         {¿}{{?`}}1
}


% Gráficos
\usepackage{pgfplots}
\pgfplotsset{width=7cm,compat=1.8} % Opciones para gráficos

% Emoticonos
\usepackage{wasysym}

%% Comandos %%
% Comandos para incluir ejemplos como listings y mostrarlos
  \newcommand{\ejemplo}[1]{\lstinputlisting{./Ejemplos/#1}}
  \newcommand{\muestra}[1]{\input{./Ejemplos/#1}}

\newcommand{\espacio}{\\~\\}           % tex.stackexchange.com/questions/11622
\newcommand{\beamer}{\texttt{beamer} } % Comando para poner 'beamer' con un estilo único.

%% Temas %%
% Tema y tema de color
\usetheme{Dresden}
\usecolortheme{dolphin}
\useinnertheme{circles}
\setbeamercovered{transparent}

% Colores de los bloques

% Bloque normal
  \setbeamercolor{block title}{bg=title,fg=links}
  \setbeamercolor{block body}{bg=backg,fg=black}
% Bloque alerta
  \setbeamercolor{block title alerted}{fg=red!70!black,bg=title!92!red}
  \setbeamercolor{block body alerted}{fg=black,bg=backg}
% Bloque ejemplo
  \setbeamercolor{block title example}{fg=green!70!black,bg=title!92!green}
  \setbeamercolor{block body example}{fg=black,bg=backg}

% Enlaces (tex.stackexchange.com/questions/13423)
  \hypersetup{colorlinks,linkcolor=,urlcolor=links}


%% Título y otros %%
\title{Cómo usar \beamer}  % Título
\subtitle{Una guía escrita en \beamer}   % Subtítulo
\author[Pablo Baeyens]{Pablo Baeyens Fernández \\ \href{mailto:pbaeyens31+github@gmail.com}{pbaeyens31+github@gmail.com}} %Autor y e-mail
\date{DGIIM} % Fecha


%% Presentación %%
\begin{document}

\frame{\titlepage}% Página de título

\frame{ % Índice
  \frametitle{Índice}
  \tableofcontents
}

\section{Introducción}

\frame{
\frametitle{¡Contribuye!}
  El código fuente de éstas diapositivas está disponible en:
\begin{center}
  \huge \href{http://github.com/pbaeyens/beamer}{github.com/pbaeyens/beamer}
\end{center}
  Erratas, correcciones y aportaciones son bienvenidas.
}


\frame{
  \frametitle{¿Qué es \beamer?}
  \beamer es una clase de documento de $\LaTeX$ que genera diapositivas
  o transparencias con animaciones, gráficos, tablas...

  Puede compilarse con casi cualquier compilador de $\LaTeX$ y personalizarse al detalle.

  %\espacio
  \pause

    \begin{block}{\texttt{pandoc}}
      \texttt{pandoc} no sirve \frownie{}. Podemos generar presentaciones con \beamer
      pero tiene \href{http://johnmacfarlane.net/pandoc/demo/example9/producing-slide-shows-with-pandoc}{una sintaxis propia}.
    \end{block}
}

\frame{
  \frametitle{Instalación}

  Para usar \beamer se necesitan 3 paquetes:

  \begin{itemize}
    \item \beamer
    \item \texttt{pgf}
    \item \texttt{xcolor}
  \end{itemize}
    \pause
    \begin{block}{Debian/Ubuntu y derivados}
    En Debian y derivados podemos instalar \texttt{latex-beamer}. También podemos utilizar \href{http://www.ctan.org/pkg/texliveonfly}{\texttt{texliveonfly}}.
  \end{block}
}

\frame{
  \frametitle{Primeros pasos}

  Para empezar a usar \beamer, indicamos la clase del documento:

    \begin{center}
      \Large \texttt
      {\color{black}\textbackslash}{\color{keywords}documentclass}{\color{black}\{beamer\}}
    \end{center}

    \pause
    \begin{alertblock}{Cambiando el tipo de letra}
      \beamer utiliza una letra sin serifa para las fórmulas matemáticas por defecto.
      Podemos utilizar la fuente de \texttt{article} pasando el argumento \texttt{mathserif}.
    \end{alertblock}
}

\frame{
  \frametitle{Diapositivas}
  Creamos las diapositivas utilizando el entorno o la función \texttt{frame}:

  \ejemplo{ejemplo1.tex}

  \pause
  \begin{block}{Tipos de diapositivas}
    \begin{itemize}[<+->]
      \item \texttt{shrink}:  Reduce el tamaño para introducir más contenido.
      \item \texttt{plain}:   Diapositiva simple, útil para imágenes.
      \item \texttt{fragile}: Necesario para mostrar \texttt{verbatim}.
      \item \texttt{allowframebreaks}: Divide el contenido en diapositivas.
    \end{itemize}
  \end{block}
}

\section{Aspecto}

%\subsection{Temas}
\subsection{Bloques}

\begin{frame}
  \beamer permite crear bloques para estructurar la información:
  \espacio
  \begin{columns}
   \column{.5\textwidth}
      \pauses
      \begin{block}{Bloques normales}
        Se crean con el entorno \\
        {\centering \texttt{block}.}
      \end{block}

      \pause
      \begin{alertblock}{Bloques alerta}
        Se crean con el entorno \texttt{alertblock}.
      \end{alertblock}
   \column{.5\textwidth}
    \pause
    \begin{exampleblock}{Bloques ejemplo}
      Se crean con el entorno \texttt{exampleblock}.
    \end{exampleblock}

    \pause
    \begin{theorem}
      No existen números mayores que 2.
    \end{theorem}
  \end{columns}

  \pause
  \espacio
  Los bloques matemáticos (teoremas, demostraciones...) tienen el mismo aspecto
  que los bloques normales.
\end{frame}

%\subsection{\textit{Overlay}}
%\subsection{Columnas}

\subsection{Tipo de letra}

\frame{
\frametitle{Tamaño}
\begin{columns}
  \column{.5\textwidth}
    Podemos cambiar el tamaño de letra utilizando los comandos habituales en $\LaTeX$.
    También podemos cambiar el color, utilizando el paquete \texttt{xcolor}.

  \column{.5\textwidth}
    \begin{itemize}
      \item \texttt{\tiny \textbackslash tiny}
      \item \texttt{\scriptsize \textbackslash scriptsize}
      \item \texttt{\footnotesize \textbackslash footnotesize}
      \item \texttt{\small \textbackslash small}
      \item \texttt{\normalsize \textbackslash normalsize}
      \item \texttt{\large \textbackslash large}
      \item \texttt{\Large \textbackslash Large}
      \item \texttt{\LARGE \textbackslash LARGE}
      \item \texttt{\huge \textbackslash huge}
    \end{itemize}
\end{columns}
}


\section{Gráficos y otras maravillas de \texttt{tikz}}

\frame{
\begin{columns}[c]
  \column{.6\textwidth}
    \begin{tikzpicture}
      \begin{axis}[
      ybar stacked,
      bar width=.7cm,
      ytick=\empty,
      symbolic x coords={\beamer,LibreOffice,PowerPoint},
      xtick=data,
      ]
        \addplot[fill=bars] coordinates
        {(\beamer,100) (LibreOffice,9) (PowerPoint,4)};
      \end{axis}
    \end{tikzpicture}

  \column{.4\textwidth}
   Podemos hacer gráficos con \href{https://ctan.org/pkg/pgfplots}{\texttt{pgfplots}} (aunque de este paquete se puede hablar tanto como de \beamer).
  \end{columns}
}
\end{document}
